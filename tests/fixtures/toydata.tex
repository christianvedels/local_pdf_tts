% This paper is entirely fictional.  It was generated by an AI (Claude Code)
% as test data for the local_pdf_tts project.  The authors, findings, and
% cited references are all made up.

\documentclass[11pt]{article}
\usepackage{booktabs}
\usepackage{graphicx}

\title{Hearing What You Write: The Case for Auditory Proofreading}
\author{Jane A.\ Smith \and John B.\ Doe}
\date{February 2026}

\begin{document}
\maketitle

\noindent\textbf{Disclaimer:} This paper is entirely fictional.  It was
generated by an AI (Claude Code) as test data for the \texttt{local\_pdf\_tts}
project.  The authors, findings, and cited references are all made up.

\bigskip

\begin{abstract}
We argue that listening to one's own writing read aloud is a powerful yet
underutilised proofreading strategy.  Drawing on cognitive psychology and
practical experience, we show that auditory review catches errors that
silent reading routinely misses.  We present a small experiment in which
participants identified 27\% more errors when proofreading by ear compared
to proofreading by eye alone.
\end{abstract}

\section{Introduction}

Every writer knows the frustration of discovering a typo in a manuscript
that has already been read five times.  Silent reading engages a predictive
process: the brain anticipates upcoming words and fills in gaps, often
glossing over errors that are hiding in plain sight.  Awkward phrasing,
repeated words, and missing articles slip through precisely because the
author knows what the sentence is \emph{supposed} to say.

Listening to the same text read aloud disrupts this autopilot.  The ear is
far less forgiving than the eye.  A clumsy subordinate clause that looks
acceptable on the page becomes immediately obvious when spoken.  Run-on
sentences reveal themselves as breathless monologues, and missing
punctuation turns crisp prose into an unintelligible stream.

In this paper we review the cognitive basis for auditory proofreading,
report a controlled experiment, and discuss practical tools that make the
technique accessible to academic writers.

\section{Background}

Research on reading comprehension has long established that skilled readers
process text in chunks rather than word-by-word.  This chunking ability,
while essential for fluent reading, can mask local errors.  Studies by
Daneman and Stainton (1993) showed that proofreading accuracy declines as
the text becomes more predictable, because the reader's expectations
override the actual input.

Auditory processing follows a different pathway.  Speech unfolds in real
time and cannot be skimmed.  The listener must process each word
sequentially, which forces attention to surface-level details that silent
reading may skip.  Several writing guides recommend reading drafts aloud,
but empirical evidence on the effectiveness of this strategy remains sparse.

Table~\ref{tab:results} summarises our experimental findings.

\begin{table}[h]
\centering
\caption{Mean errors detected per 1{,}000 words by proofreading method.}
\label{tab:results}
\begin{tabular}{lrrr}
\toprule
Method & Spelling & Grammar & Style \\
\midrule
Silent reading  & 4.2 & 2.8 & 1.1 \\
Reading aloud   & 5.1 & 3.9 & 2.4 \\
Listening (TTS) & 5.3 & 4.1 & 3.0 \\
\bottomrule
\end{tabular}
\end{table}

\section{Results}

Participants who listened to a text-to-speech rendering of their own
manuscripts detected significantly more errors than those who proofread
silently ($p < 0.01$).  The advantage was most pronounced for stylistic
issues --- sentence rhythm, word repetition, and convoluted syntax ---
where the listening group found nearly three times as many problems.

Spelling errors showed the smallest difference between conditions, which
is expected: spell-checkers already catch the majority of such mistakes.
Grammar errors occupied an intermediate position.  Interestingly, several
participants reported that hearing their writing made them aware of tonal
inconsistencies they had never noticed before.

\section{Conclusion}

Listening to one's own writing is a simple, effective, and surprisingly
enjoyable way to improve manuscript quality.  Modern text-to-speech tools
make it easy to convert a draft into audio and review it during a walk or
commute.  We encourage researchers to incorporate auditory proofreading
into their revision workflow --- the ears catch what the eyes forgive.

\end{document}
