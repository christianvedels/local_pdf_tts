% Mock main LaTeX file for the test Overleaf project.
% This paper is entirely fictional and was generated as test data.

\documentclass[12pt]{article}
\usepackage{booktabs}
\usepackage{graphicx}
\usepackage{natbib}

\title{Economic Returns to Education: A Cross-Country Analysis}
\author{Maria Garcia \and Thomas Schmidt}
\date{February 2026}

\begin{document}

\maketitle

\noindent\textbf{Disclaimer:} This paper is entirely fictional and was
generated as test data for the \texttt{local\_pdf\_tts} project.

\begin{abstract}
We examine the economic returns to education using panel data from
42 countries over the period 1990--2020.  Exploiting variation in
compulsory schooling laws as an instrument for educational attainment,
we find that an additional year of schooling raises wages by
approximately 10\% on average.  Returns are higher in high-income
countries and lower in low-income countries, consistent with
complementarity between education and physical capital.  Our results
are robust to alternative sample restrictions and measures of
educational attainment.
\end{abstract}

\section{Introduction}

Education is widely regarded as a key driver of economic growth and
individual prosperity.  Since the seminal work of \citet{mincer1974},
economists have devoted considerable effort to measuring the private
return to schooling---the percentage increase in wages associated with
one additional year of education.

Estimating this return is complicated by the endogeneity of schooling
decisions.  Individuals who invest more in education may differ from
those who invest less along dimensions that are correlated with wages
but unobserved by the researcher.  Failure to account for these
differences leads ordinary least squares (OLS) estimates to overstate
or, in some cases, understate the true causal effect.

We address this challenge using variation in compulsory schooling
laws across countries and over time as an instrument for individual
educational attainment, following the approach of \citet{card1999}.
This strategy allows us to identify the local average treatment effect
of schooling on wages for compliers---individuals whose schooling
decisions were affected by changes in compulsory education policy.

\section{Data and Methodology}

\subsection{Data Sources}

We combine three publicly available datasets.  Wage and employment
data come from the International Labour Organization (ILO) harmonised
microdata covering 42 countries between 1990 and 2020.  Educational
attainment data are drawn from the Barro-Lee dataset, which provides
five-year estimates of average years of schooling for 146 countries.
Information on compulsory schooling laws is taken from a newly
assembled database compiled by \citet{psacharopoulos2004}.

The final analysis sample consists of 1{,}260 country-year
observations.  Summary statistics are presented in a supplementary
table available from the authors upon request.

\subsection{Empirical Strategy}

Our baseline specification is a standard Mincerian earnings equation
augmented with country and year fixed effects.  We estimate the model
by OLS, by two-stage least squares (IV) using compulsory schooling
laws as instruments, and by the within-group (fixed effects) estimator
that controls for time-invariant individual heterogeneity.

Table~\ref{tab:main} presents the main results.

% Main results table for the mock Overleaf project.

\begin{table}[h!]
\centering
\caption{Returns to Education by Country Income Group}
\label{tab:main}
\begin{tabular}{lccc}
\toprule
Country Group & OLS & IV & Fixed Effects \\
\midrule
High Income   & 0.12 & 0.14 & 0.11 \\
Middle Income & 0.09 & 0.11 & 0.08 \\
Low Income    & 0.06 & 0.09 & 0.05 \\
\bottomrule
\end{tabular}

\textit{Notes:} Each cell reports the estimated return to one additional
year of schooling on log wages.  Standard errors clustered at the country
level are in parentheses (not shown).  *** $p<0.01$, ** $p<0.05$,
* $p<0.10$.
\end{table}


\section{Results}

Countries at higher income levels exhibit substantially larger returns
to schooling than their lower-income counterparts.  In high-income
countries the IV estimate stands at 0.14, implying that one additional
year of schooling raises wages by 14\%.  In low-income countries the
corresponding figure is only 0.09.  This gradient is consistent with
the hypothesis that the productivity of education is complementary to
physical capital, which is more abundant in richer economies.

The gap between OLS and IV estimates is positive but modest: the IV
coefficient exceeds the OLS coefficient by about 0.02 across all
country groups.  This pattern is consistent with the presence of
ability bias that is roughly offset by measurement error in educational
attainment, a finding in line with the broader literature surveyed by
\citet{card1999}.

\section{Conclusion}

This paper provides new cross-country evidence on the economic returns
to schooling.  Using panel data from 42 countries and an instrumental
variables strategy based on compulsory schooling laws, we estimate that
an additional year of education raises wages by approximately 10\% on
average.  Returns vary systematically with the level of economic
development, being highest in high-income and lowest in low-income
countries.

Our findings have implications for education policy.  In countries
where returns are high and educational attainment is low, increasing
school-leaving ages or investing in secondary education may generate
substantial wage gains.  Future research should examine whether these
aggregate patterns conceal important heterogeneity across demographic
groups and sectors of the economy.

\appendix

% Appendix for the mock Overleaf project.

\section{Robustness Checks}
\label{app:robustness}

This appendix presents additional robustness checks for the results
reported in the main text.  We re-estimate the baseline specification
using alternative sample restrictions, different measures of educational
attainment, and an extended set of control variables.

\subsection{Alternative Sample Restrictions}

Table~\ref{tab:robustness} replicates the main findings when we
restrict the sample to countries with at least twenty years of
continuous data.  The point estimates remain stable across all three
estimators, suggesting that sample composition does not drive our
results.

\begin{table}[h!]
\centering
\caption{Robustness: Restricted Sample}
\label{tab:robustness}
\begin{tabular}{lccc}
\toprule
Restriction & OLS & IV & Fixed Effects \\
\midrule
Baseline (42 countries)  & 0.12 & 0.14 & 0.11 \\
Balanced panel (28 countries) & 0.11 & 0.13 & 0.10 \\
OECD only (22 countries) & 0.10 & 0.12 & 0.09 \\
\bottomrule
\end{tabular}

\textit{Notes:} See notes to Table~\ref{tab:main}.
\end{table}

\subsection{Alternative Measures of Education}

We also replace average years of schooling with the share of the adult
population that has completed secondary education.  The qualitative
conclusions are unchanged: returns remain positive, statistically
significant, and larger when identified by instrumental variables than
by OLS.


\bibliographystyle{apalike}
\bibliography{references}

\end{document}
